\poemtitle{Radbod}
\begin{verse}
In noorderwoude, koel en stil, \\ 
\ \ \ Op altare van mos gebouw \\ 
Het die Sakse hul Dondergod gedie \\ 
Met ander gode, ’n stuk of tien, \\ 
\ \ \ En hul feeste daar gehou. \\ 
\end{verse}

\begin{verse}
Maar toe kom Gods Hamer; die woude dreun \\ 
\ \ \ Soos hij preek met swaard en met vuur, \\ 
En in naam van Christus die kinders vermoor \\ 
\ \ \ En hele lande laat smeul en laat smoor \\ 
Onder sang van Gods lof deur ’n monnikekoor, \\ 
\ \ \ En verwoesting van skut en van skuur; \\ 
\end{verse}

\begin{verse}
Want die naam van die Koning van Vrede dien vaak \\ 
\ \ \ Als ’n leuse vir die wat strij \\ 
Vir eer, of vir goud en die lus om te buit, \\ 
Of die mag van ’n grensende volk te stuit \\ 
\ \ \ Of om rang in die volkere rij. \\ 
\end{verse}

\begin{verse}
Die Hamer moes stadig aan swig vir die kruis, \\ 
\ \ \ Die grense van Rome word wijd, \\ 
En Radbod, die dienaar van Wodin moes buig \\ 
En sij hoof voor die God van die heerser neig \\ 
\ \ \ Stil in deemoedigheid. \\ 
\end{verse}

\begin{verse}
Lugters skijn hel deur die wierook wat walm, \\ 
\ \ \ Die bekken en water staan klaar; \\ 
Woes klink die sang van die priester-skaar \\ 
Benouwd is die saal en die rook wat waar, \\ 
\ \ \ Dan stijg weer die jubelpsalm. \\ 
\end{verse}

\begin{verse}
Daar nader Radbod in witte gewaad, \\ 
\ \ \ Sij o\"e blij strak op die grond. \\ 
Die doopformulier is al afgedaan \\ 
Een voet in die water reeds, blij hij staan \\ 
\ \ \ En ’n glimlag speel om sij mond; \\ 
\end{verse}

\begin{verse}
``Priester, jij sê ’k sal die hemel be\"erf, \\ 
\ \ \ Maar ver is die poorte en nouw: \\ 
Waar is mij vaders dan, die oor die see \\ 
Gevaar het, en alles daarvan gevee \\ 
En kuste vervul het met wanhoop en wee, \\ 
\ \ \ Waar rus mij vaadre nou?'' \\ 
\end{verse}

\begin{verse}
``In die hel!'' sis die priester. En Radbod trap trug \\ 
\ \ \ Van die bekken en hef homself fier \\ 
Omhoog met o\"e wat bliksem van trots \\ 
\ \ \ Soos van arend of gier: \\ 
\end{verse}

\begin{verse}
``Priester, jouw hemel kan jij maar hou; \\ 
\ \ \ Liefs daar dan met graamtes van christne daarbo, \\ 
Te gierig te eet, net ruim om te glo. \\ 
In Valhalla met Wodin en Thor, \\ 
Tot Ragnarok die heelal soos ’n deken vouw, \\ 
Sal ek fees -- tot die see\"e verdor.'' \\ 
\end{verse}

\begin{verse}
In die saal van Wodin gloei dit, \\ 
Uit die goue bekers vloei dit, \\ 
En in vreugd en heildronk loei dit: \\ 
\end{verse}

\begin{verse}
\ \ \ \ \ \ \ \ \ \ \ \ \ \ \ \ \ \ \ \ \ \ \ \ \ \ \ \ \ \ \ \ \ \ \ ``Dondergod, \\ 
Ons sal lij en strij met Wodin, \\ 
Vir Valhalla en ons Godin, \\ 
Wee die Vredekoning, snood in \\ 
\ \ \ \ \ \ \ \ \ \ \ \ \ \ \ \ \ \ \ \ \ \ \ \ \ \ \ \ \ \ \ \ \ \ \ Sij gebod! \\ 
In die rook van slagting leef ons \\ 
Vir die eer van oorlog streef ons \\ 
En met jouw gedij of sneef ons, \\ 
\ \ \ \ \ \ \ \ \ \ \ \ \ \ \ \ \ \ \ \ \ \ \ \ \ \ \ \ \ \ \ \ \ \ \ Dondergod!'' \\ 
\end{verse}