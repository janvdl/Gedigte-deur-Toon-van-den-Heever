\poemtitle{Rit-rijmpie}
\begin{verse}
Die koe\"els die blits, dat die klippe so brand -- \\ 
\ \ \ \ \ \ \ \ Hoera vir die reuk van ’n roer! -- \\ 
Maar die tien en Jan Pieterse hou daardie nek, \\
\ \ \ \ \ \ \ \ Help bewaar ou Transvaal vir die Boer! \\
\end{verse}

\begin{verse}
``’n Uur nog, net ene'' -- ’n uur van die hel! --  \\
\ \ \ \ \ \ \ \ ``Als ons maar die nek kan hou, \\
Dan is ons kommando die drif veilig deur!'' \\
\ \ \ \ \ \ \ \ En hij bid sonder handegevouw! \\
\end{verse}

\begin{verse}
Wild en wijd, wild en wijd, \\ 
\ \ \ \ \ \ \ \ Gee die perde maar teuel en spoor; \\ 
Jij hoor net daaronder die hoewegedonder, \\ 
\ \ \ \ \ \ \ \ En hierbo fluit die koe\"els langs jouw oor. \\ 
\end{verse}

\begin{verse}
Daar hinnik ’n liddiet, rijperd van die Dood \\ 
\ \ \ \ \ \ \ \ Blits hemel en aarde dooreen; \\ 
Die stof wijk -- die mense ’s die drif veilig deur --  \\ 
\ \ \ \ \ \ \ \ Bij Jan Pieterse leef nog net een. \\ 
\end{verse}

\begin{verse}
``Ons trap!'' kommandeer hij. Wild en wijd \\ 
\ \ \ \ \ \ \ \ Gee hul perde maar teuel en spoor; \\ 
Hij hoor net daaronder die hoewegedonder \\ 
\ \ \ \ \ \ \ \ En ’n skreeuw -- hij ’s allenig nou oor. \\ 
\end{verse}

\begin{verse}
Trug nou, trug deur die vlaag van lood, \\ 
\ \ \ \ \ \ \ \ Sij maat kan hij daar niet laat blij! \\ 
Al om hom daar gons dit die lied van die dood, \\ 
\ \ \ \ \ \ \ \ Waar sij sikkel al snerpende snij. \\ 
\end{verse}

\begin{verse}
Met die wonde man dwars oor sij saalboom geleg \\ 
\ \ \ \ \ \ \ \ Gaat hij stadig en stappende heen, \\ 
Geen teken van haas, hij rij peinsende voort \\ 
\ \ \ \ \ \ \ \ Op ’n stap deur die koe\"el-gereen. \\ 
\end{verse}

\begin{verse}
``Laat staan, hij’s ’n held -- Three cheers’ vir die Boer!'' \\ 
\ \ \ \ \ \ \ \ Sê die Engelsman, hees in die keel. \\ 
Die bulte weerklink met vier harde hoera’s, \\ 
\ \ \ \ \ \ \ \ En die lug word met helmette geel! \\ 
\end{verse}

\begin{verse}
Jan Pieterse draai hom sij vijande toe \\ 
\ \ \ \ \ \ \ \ En wuif hulle dank met sij hoed; \\ 
Al stappend verdwijn hij op die horison daar \\ 
\ \ \ \ \ \ \ \ In ’n raam van die westergloed. \\ 
\end{verse}

\begin{verse}
So hoera, vir die klank van stiebeuel en spoor \\ 
\ \ \ \ \ \ \ \ En hoera, vir die reuk van ’n roer! \\ 
God behoede ons veld van die vreemde geweld \\ 
\ \ \ \ \ \ \ \ En bewaar ons ou land vir die Boer. \\ 
\end{verse}