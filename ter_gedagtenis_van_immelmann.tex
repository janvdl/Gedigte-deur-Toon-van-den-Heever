\poemtitle{Ter gedagtenis van Immelmann}
\begin{verse}
Is Icarus gevalle, waarom huil die wind so sag? \\ 
\ \ \ \ \ \ \ Waarom strooi die meerminkore perels neer? \\ 
Hoor jij nie daar’s geen gelispel, net ’n droewe golweslag \\ 
\ \ \ \ \ \ \ En die nautilus pluik nou g’n purper meer.
\end{verse}

\begin{verse}
Is Icarus gevalle in die vreemde, ver van huis, \\ 
\ \ \ \ \ \ \ Waar ons eie bulte nooit sij graf sal merk, \\ 
Maar in beddings van korale waar die skulpe ewig ruis, \\ 
\ \ \ \ \ \ \ Sij hoof geskraag op ’n gewrakte vlerk?
\end{verse}

\begin{verse}
Is Icarus gevalle? Ja, hij kon ’n la\"e swerk \\ 
\ \ \ \ \ \ \ Nooit vir sij ho\"e moed genoegsaam vind, \\ 
En stijgend steeds van kreits tot kreits met kloek gespanne vlerk \\ 
\ \ \ \ \ \ \ Trotseer hij altijd wakker weer en wind,
\end{verse}

\begin{verse}
Tot sfere waar die vrijheids son sij strale nijdig hef \\ 
\ \ \ \ \ \ \ Met wimpers in die weste rooi omkrans, \\ 
Toe word meteens sij wieke met Apol s’n pijl getref, \\ 
\ \ \ \ \ \ \ Virseker was die val met sterreglans.
\end{verse}

\begin{verse}
Ja, Icarus gevalle! maar hij wou g’n slaaf meer leef, \\ 
\ \ \ \ \ \ \ Net wie nooit wil stijg kan moeilik immer val, \\ 
Strooi blomme oor die water, laat die maagde kransies weef, \\ 
\ \ \ \ \ \ \ Wijl die water droewig weeklaag teen die wal.
\end{verse}

\begin{verse}
Ja, Icarus gevalle! Maar solang die winde suis, \\ 
\ \ \ \ \ \ \ Solang die seeskulp fluister van die see, \\ 
Sal vertel word van die ridder wat gevel is ver van huis, \\ 
\ \ \ \ \ \ \ Vir vrijheid en vir glorie, lang gelee!
\end{verse}