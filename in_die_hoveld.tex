\poemtitle{In die Hoveld}
\begin{verse}
In die Hoveld, waar dit oop is en die hemel wijd daarbo, \\ 
\ \ \ \ \ Waar kuddes waaigras huppel oor die veld, \\ 
Waar ’n mens nog vrij kan asemhaal en aan ’n God kan glo, \\ 
\ \ \ \ \ Staat mij huisie, wat ek moes verlaat vir geld. \\ 
En als ek in die gange van die mijn hier sit en droom \\ 
\ \ \ \ \ Van die winde op die Hoveld, ruim en vrij, \\ 
Dan hoor ek die geklinkel van mij spore saal en toom, \\ 
\ \ \ \ \ Sawends als ek bees of skaap toe rij. \\ 
\end{verse}

\begin{verse}
Op die Hoveld, waar dit wijd is, waar jij baja ver kan sien, \\ 
\ \ \ \ \ (Die eilblouw bring ’n knop dan in jouw keel) \\ 
Staat mij huisie nog en wag vir mij, wag al ’n jaar of tien, \\ 
\ \ \ \ \ Waar die bokkies op die lei grafstene speel. \\ 
Maar als die tering kwaai word en ek hoor die laatste fluit, \\ 
\ \ \ \ \ Dan sweef ek naar die Hoveld op die wind \\ 
En soek dan in die maanlig al die mooiste plekkies uit \\ 
\ \ \ \ \ Waar ’k kleiosse gemaak het als ’n kind. \\ 
\end{verse}