\poemtitle{In die park}
\begin{verse}
Jij sê ek’s mal omdat ek dag na dag \\ 
Hier in die park allenig sit. Ou maat, \\ 
Strij help nou niks, ek sien dit in jouw oë, \\ 
En in die oë van die kindertjies, \\ 
Wat bang-meelijdend naar mij kijk en dan \\ 
Weer agter rok van Ma of meid induik. \\ 
Asjoublieftog, neef, wag net ’n beetjie, hoor. \\ 
Vir daë sit ek hier alleen en lag, \\ 
En tel die groewe wat al dieper snij \\ 
In mij gesig, soos in ’n pad waar waans \\ 
Met sware vragte spore agter laat; \\ 
Nou moet ek praat en jij moet luister, neef, \\ 
Moet help die vragte biekie ligter maak \\ 
Of ek gaan vasval. Dis reg, luister nou. \\ 
\end{verse}

\begin{verse}
Kijk daardie grasvlek, waar die kinders speel \\ 
En waar die gieter sprei; kijk die kwikkie daar, \\ 
Kom nader tot die waterstroom hom rol \\ 
Oor die kweek, dan kom hij weer, en nou \\ 
Sit hij sij bors en krap. Uit die vijwerkom \\ 
Spuit slank ’n waterstraal omhoog en sprei \\ 
In duisend perelsnoere, soos ’n maagd \\ 
Wat trag haar liggaam met haar haar te dek. \\ 
Hoor hoe die water lag, net soos ’n vrouw \\ 
Wat haar kindjie opgooi en weer veilig vang; \\ 
Kijk die blommemure hier, en daar, en daar, \\ 
Soos strome louter goud en westergloor \\ 
Waaruit die bij-gesoem mij rede sus, \\ 
(Jij lag om ek sê rede) en die heuningbek \\ 
Bewegingloos al oor die blomme tril, \\ 
(Selfstandig soos die hoogste Englekoor), \\ 
Of blink en groen van bed tot bedding straal. \\ 
Kijk die kinders, man, verstaan jij wat ek meen? \\ 
Verstaan jij dat ons almaal, iedereen, \\ 
’n Engel het, moet hê, wat ons bewaar? \\ 
Hoe sal ek jou tog maak verstaan? Let op! \\ 
Saands in mij bed steek ek ’n sigaret \\ 
Aan, elke nag, voor ek die lig uitdraai; \\ 
En onder bij mij voete, teen die muur \\ 
Hang daar ’n prent van Echo, dit wil sê \\ 
’n Prent van ’n beeld van Echo, slank en wit, \\ 
Wat wag en wag en luister, ewig wag; \\ 
Virewig en verniet. Dan glim die kool \\ 
Van mij si’gret in die prent, en ek denk so: \\ 
``Narcissus kom tog nooit, sal nimmer kom; \\ 
``Maar al sal Echo nooit haar minnaar sien, \\ 
``Haar tweeling suster Weerlig sal sij kus \\ 
``En die refleksie van ’n vonk omhels; \\ 
``Want Weerlig moet vir Weerklank troos, of ek \\ 
``Word gek, die spanning is te groot, net soos \\ 
``Die blomme teen die muur ’n siekmens pla.'' \\ 
Nou, so het ek en jij en iedereen \\ 
’n Engel wat ons wandel gadeslaan; \\ 
Ek hoor sij wiekgeruis in kinderlag \\ 
En sien sij vleu’els blits in manestraal, \\ 
En in die lag van maagde, en sij stem \\ 
In water wat so sag teen walle klots. \\ 
\end{verse}

\begin{verse}
Toe ek nog klein was het mij moeder steeds, \\ 
Als sij mij naar mij bed bring en laat bid, \\ 
Vertel van hierdie wagter, dus geen vrees. \\ 
En deur mij jeugd in lig of duisternis \\ 
Het ek geen trouwer metgesel gehad \\ 
’n Wese wat oor siel en liggaam heers, \\ 
’n Afgesant van God. \\ 
\end{verse}

\begin{verse}
\ \ \ \ \ \ \ \ \ \ \ \ \ \ \ \ \ \ \ \ \ En toe een aand, \\
Ek sien dit nog, die kerk was stapelvol, \\ 
En hel het lugterkroon en lamp gestraal, \\ 
En orgel-tone triomf uitbasuin. \\ 
Die predikant was glad van wang en stem, \\ 
-- Ek sien die lig nog op sij voorhoof glim --  \\ 
En wijl toehoorders weeldrig sit en gril, \\ 
Het hij van voorbestemming sag vertel, \\ 
(Predestinasie, wat jij dit wil noem) \\ 
In periode ruim en afgerond. \\ 
\end{verse}

\begin{verse}
Die wee, mij God, ek voel dit nog, die wee \\ 
Waarmee ek daardie kerkgebouw verlaat het; \\ 
Dit het soos vure oor mij siel geskroei, \\ 
Soos wurme in mij murg geboor, mij brein \\ 
Laat brand soos koors. \\ 
\ \ \ \ \ \ \ \ \ \ \ \ \ \ \ \ \ \ \ \ \ Waarom, dag ek, waarom \\ 
Die mensdom selfbewus gemaak, gebied \\ 
’n Ewig voortbestaan, totdat hij lus \\ 
En toe, net soos ’n kind ’n hondjie terg, \\ 
Dit weggeruk? Die glorie, adel, prag, \\ 
Die wilde skoonheid van ons groot heelal, \\ 
Wat soos ’n riet mij hele siel laat tril, \\ 
Als die grootse harmonie daaroor gaat ruis, \\ 
Gaat alles swanger aan verderf en hel; \\ 
Wijl Hij, die maker, homself Liefde noem! \\ 
\end{verse}

\begin{verse}
``En daar ek so ’n God nie kan bemin, \\ 
``Is ek verdoemd,'' hoor ek ingewing sê; \\ 
Van toe af het ’k mij engel meer bemin \\ 
Dan moeder ooit haar één gebreklik kind. \\ 
Ek sien die weedom op sij voorhoof rus \\ 
Soos donderwolke op die bergekam, \\ 
Waaruit sij oog met dowwe meelij blits; \\ 
Ek sien sij wieke druip soos ridderpluim \\ 
Van slaë slap en hooploos strijë moeg. \\ 
Verbeel jou, man, dis voorbestem en voor \\ 
Die grondlegging der wereld al besluit \\ 
Dat ek verdoemd is, hij oor mij moet waak; \\ 
Wat hij ooit doen mog, of ek ooit verrig, \\ 
Die skrif staat vas en pal. \\ 
\ \ \ \ \ \ \ \ \ \ \ \ \ \ \ \ \ \ \ \ \ Verstaan? Ek sien \\ 
Die skaar van Engelwagters wat ons hoed, \\ 
Ons, van wie die merendeel verdoemd is, \\ 
Want sê die skrif nie so? Mij Engel nou \\ 
Moet mij bemin, en altijd sien en sorg, \\ 
Daar hij ’n ruimer siel het, dieper voel \\ 
Die wanhoop van sij taak. \\ 
Als storme dreig sien jij die vooltjies skuil \\ 
En hoor hul swijg, die aandblom sluit homself \\ 
En snare kan gelijke snare roer: \\ 
Die weedom van mij Engel rus op mij \\ 
En maak die groewe dieper; maar als ek weer \\ 
Gaan denk aan sij meelije, trouwe sorg, \\ 
Hoor ek sij vleuels ritsel in die gras, \\ 
Sij wieke-slae pols dan in mij oor, \\ 
\ \ \ \ \ Sij wondre liefde maak mij siel dan week. \\ 
Dis dan wat ek hier sit en lag en sien \\ 
Hoe dat die kinders stoei, die meide slaap \\ 
En sonlig dartel met die druppels speel; \\ 
Dis alles suiwer weelde, alles vol \\ 
Met die liefde van mij Engel. \\ 
\ \ \ \ \ \ \ \ \ \ \ \ \ \ \ \ \ \ \ \ \ Die mense sê \\ 
Ek’s gek, kranksinnig, of in kindertaal, \\ 
Mij varkies uit die hok. Nog als student \\ 
Het ek geleer van Plato en sij grot \\ 
En hoe die skaduwlanders sieners terg; \\ 
Nou, ek sien Engle, voel hul weedom, weet \\ 
Dat dit om ons is wat hul treur. Weet jij \\ 
Of wat jij sien die son is? Dankie neef, \\ 
Die bitterheid des doods het nou gewijk, \\ 
Ek sal weer kijk hoedat die kwikkies rol, \\ 
En heuningbekkies blits van blom tot blom \\ 
En bommelbije deur die beddings gons \\ 
Totdat die nag dit alles immer dek. \\ 
\end{verse}
