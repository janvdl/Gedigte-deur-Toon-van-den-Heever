\poemtitle{En alles word jou toegeworpe?}
\begin{verse}
Op sij troon sit die simpele koning van Frankrijk en lag \\ 
Met wijdblouwe o\"e, so leeg soos die van ’n kind, \\ 
Toe hoor hij iets dreun soos die see; en met skrik kom sij wag \\ 
Vertel van ’n magtig gejubel, gedra op die wind: \\ 
\end{verse}

\begin{verse}
\ \ \ \ \ \ \ \ \ \ \ \ \ \ \ \ \ \ ``Ho\"er, ho\"er \\ 
\ \ \ \ \ \ \ \ \ \ \ \ \ \ \ \ \ \ Op jul skilde, \\ 
\ \ \ \ \ \ \ \ \ \ \ \ \ \ \ \ \ \ Hef Pepijn die Korte ho\"er, \\ 
\ \ \ \ \ \ \ \ \ \ \ \ \ \ \ \ \ \ Geef ’n held vir ons tot koning, \\ 
\ \ \ \ \ \ \ \ \ \ \ \ \ \ \ \ \ \ Die kan heers in fees of veld!  \\ 
\end{verse}

\begin{verse}
\ \ \ \ \ \ \ \ \ \ \ \ \ \ \ \ \ \ Luider, vrijer \\ 
\ \ \ \ \ \ \ \ \ \ \ \ \ \ \ \ \ \ Galm ons juigkreet, \\ 
\ \ \ \ \ \ \ \ \ \ \ \ \ \ \ \ \ \ Laat die heidenvorste sidder, \\ 
\ \ \ \ \ \ \ \ \ \ \ \ \ \ \ \ \ \ Nou het ons ’n held tot heerser: \\ 
\ \ \ \ \ \ \ \ \ \ \ \ \ \ \ \ \ \ Christus preek ons met geweld!'' \\ 
\end{verse}

\begin{verse}
Maar in sij paleis sit ’n simpele koning en glimlag \\ 
En sij o\"e was blouw en onskuldig, soos die van ’n kind; \\ 
Hij sit daar en droom van Madonnas, en wat of die wag \\ 
Kom vertel van ’n magtig gejubel gedra op die wind? \\ 
\end{verse}