\poemtitle{Die possie}
\begin{verse}
Als die donker ’n reisiger mog bespring \\ 
Daar waar die witpad mijl aan mijl, \\ 
Virbij die Possie, die ou bouwval, \\ 
Van die maan belig oor die vlakte peil; \\ 
Dan sien hij deur vensters, wat dof staan en staar, \\ 
Hoe die maanlig die balke laat rouwstrepe trek \\ 
Oor die vloer, waar net distels en klitse verjaar; \\ 
Net die uil word gehoor -- so lang gerek. \\ 
Die reisiger haas hom dan, knoop toe sij jas, \\ 
Gee die perd fluks ’n raps, wip die driffie deur, \\ 
En adem verlig -- want die boere beweer \\ 
Die Possie ’s nie pluis, en waens val vas \\ 
In die droogste tijd, en die perde snork \\ 
Als jij eensaam daar in die maanlig rij --  \\ 
En saands sien jij hierlangs ’n stofstreep staan \\ 
Soos die bees- en die skaapwagters huistoe snij. \\ 
\end{verse}

\begin{verse}
Die oumense fluister van eg sonder liefde, \\ 
(En knik naar die Possie), van min sonder trouw; \\ 
Van die tragiese ``driehoek'' -- dis lankal gebeur, \\ 
Dis alles virbij, al die skrik, al die rouw. \\ 
Hoor, als die biesies so knikkend, knikkend fluister \\ 
En die vleiwind, bedwelm, van die aandblom swaar; \\ 
Hoor jij dan niks als jij stil gaat staan en luister, \\ 
Wat die lispel als beteken, al die vreemde gebaar? \\ 
\end{verse}

\begin{verse}
\ \ \ \ \ \ \ \ \ ``Dis lewes en lewes gelede \\ 
\ \ \ \ \ \ \ \ \ \ \ \ En jare van swerwe, \\ 
\ \ \ \ \ \ \ \ \ Toe word dit daarbinne gebede \\ 
\ \ \ \ \ \ \ \ \ \ \ \ Vir iemand wat sterwe. \\ 
\end{verse}

\begin{verse}
\ \ \ \ \ \ \ \ \ --, Vra om genade, mij broeder, \\ 
\ \ \ \ \ \ \ \ \ \ \ \ Die dood is nabij’ --  \\ 
\ \ \ \ \ \ \ \ \ Bij die grote Alheerser, Alhoeder \\ 
\ \ \ \ \ \ \ \ \ \ \ \ Gena nie vir mij! \\ 
\end{verse}

\begin{verse}
\ \ \ \ \ \ \ \ \ Sij ’t gevlij: Ag spaar net mij lewe, \\ 
\ \ \ \ \ \ \ \ \ \ \ \ Ek sal weggaan van hier, \\ 
\ \ \ \ \ \ \ \ \ In mij arme gesidder en bewe \\ 
\ \ \ \ \ \ \ \ \ \ \ \ Daar bij die rivier. \\ 
\end{verse}

\begin{verse}
\ \ \ \ \ \ \ \ \ Wat skijn daar so wit op die water \\ 
\ \ \ \ \ \ \ \ \ \ \ \ Daar onder die bome? \\ 
\ \ \ \ \ \ \ \ \ Ek wens dis al oor! Is daar later nog \\ 
\ \ \ \ \ \ \ \ \ \ \ \ Nog drome?'' \\ 
\end{verse}

\begin{verse}
En daarom so stil bij die wilkerboom, \\ 
Dit fluister so saggies, dit roer skaars die lug; \\ 
Die water is stil daar ’n oomblik, en vlug \\ 
Dan fluistrend en lispelend voort op die stroom. \\ 
\end{verse}
\attrib{(Met vergunning van ``De Burger'').}