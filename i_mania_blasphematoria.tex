\poemtitle{I. Mania Blasphematoria}
\begin{verse}
Daarbo sit die Gode en feesvier \\ 
\ \ \ \ \ Met nektar en ambrosijn, \\ 
En Zeus en Apol, met ’n hees lier, \\ 
\ \ \ \ \ Stoot aan met die ou Juppijn: \\ 
Verroes nou is Ceres haar sikkel \\ 
\ \ \ \ \ Minerva die mijmer en gaap, \\ 
In die arme van Venus gewikkel \\ 
\ \ \ \ \ Lê Allah en slaap. \\ 
\end{verse}

\begin{verse}
Net Mars word vermis uit die howe, \\ 
\ \ \ \ \ Waar vreugd word ontvang en gegee: \\ 
Van die wereld stijg klanke naar bowe, \\ 
\ \ \ \ \ Die geween van ’n mensdom in wee. \\ 
Nou klink dit nog we\"er, maar weker: \\ 
\ \ \ \ \ ``Geef redding, geef uitkoms o Here!'' \\ 
In antwoord klink beker aan beker: \\ 
\ \ \ \ \ ``Musiek van die sfere!'' \\ 
\end{verse}

\begin{verse}
Soos ’n perkament rol skroei die hemel \\ 
\ \ \ \ \ En die son word gedoof, en die maan \\ 
Kijk siek op ’n magtig gewemel, \\ 
\ \ \ \ \ Waar dit volke aan volke staan; \\ 
Star en wijdogig van wonder \\ 
\ \ \ \ \ En stil staan dit, skare aan skaar, \\ 
En verbluf staan die Gode daaronder \\ 
\ \ \ \ \ Op hul regters te staar. \\ 
\end{verse}

\begin{verse}
En nou word die boeke geopen, \\ 
\ \ \ \ \ Wat getuie, hoe eeuw na eeuw \\ 
Die aarde in bloed was gedoop en \\ 
\ \ \ \ \ Verniet naar die Gode geskreeuw: \\ 
Voorbestemming het almaal gebonde, \\ 
\ \ \ \ \ Door omgewing was ieder vervoer --  \\ 
Dan verg dit nog lof uit die monde \\ 
\ \ \ \ \ Van Belg of van Boer. \\ 
\end{verse}

\begin{verse}
Star, als wijdogig van wonder, \\ 
\ \ \ \ \ Staan die regters daar, rij aan rij, \\ 
En die heersers die sidder daaronder, \\ 
\ \ \ \ \ Kan die blik van triljoene nie lij, \\ 
Smeek beskerming van heuwels en storme, \\ 
\ \ \ \ \ Van ’n wereld, wat weier te dra, \\ 
En wroeg in die slijk soals worme \\ 
\ \ \ \ \ En vlei om gena. \\ 
\end{verse}