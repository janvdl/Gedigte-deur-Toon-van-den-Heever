\poemtitle{Die laatste Trekker}
\begin{verse}
Wat baat dit dat ek thuis blij sit en tel \\ 
Hoe gouw die sonne weswaarts rol en sink, \\ 
En hoe mij kuddes meer word, wijl mij hand \\ 
Sij vasheid afleer en mij oog verdof? \\ 
Die sonne sink daar deur ’n purper poort, \\ 
Waaruit die onbekende lok: mij bloed \\ 
Bruis harder; want dis eeuw wat roep tot eeuw, \\ 
En vlaktes tot die spruit van trekker-vaders, \\ 
Wat oorkant daardie blouw groot-water eers \\ 
Die mag van Rome het geknak, hul naam \\ 
’n Bijwoord, waar die suider-moeders mee \\ 
Hul kroos in slaap gesus het. Weer daarna, \\ 
Toe daar die lug onrein word, alles mak, \\ 
Net dwinglandij en onreg woes floreer, \\ 
Toe het die vrij\"e borste met die swaals \\ 
Die wieke suidwaarts uitgeslaan, geswerf, \\ 
Geworstel deur die woesternij, gewin, \\ 
En hier die trekker ras weer oorgeplant. \\ 
\end{verse}

\begin{verse}
Soos knapies eers die koring strooi, eers vrij \\ 
Die vogele des hemels voer, dan gras \\ 
Met voolint en met lijm bestrijk, en ruil \\ 
Die vrijheid vir hul gawe: ons ou land \\ 
Was rijk besaai met goud en eelgesteent. \\ 
Vervloek sij goud! Die skepper het die res \\ 
Van die metale blank en rein gemaak, \\ 
En Hel het toe die geelste Nijd gebaar. \\ 
\end{verse}

\begin{verse}
Ek hoor die trekkerswelpe word nou mak, \\ 
En praat die taal van die veroweraar, \\ 
En groei in aansien en in rijkdom, eer; \\ 
Maar of die hok verguld is, of te nie, \\ 
En of die koring dik daarbinne lê, \\ 
Die lied ween smagtend vir die ope lug, \\ 
Waarin die bome wieg, die windjie ruis. \\ 
Span in, ons laat die purper poorte links: \\ 
Daar ruis die see, en woed, en skuim, en klink \\ 
In ewig ketting-klank van slawernij; \\ 
Laat staan die see, ons vloek. Ek hoor daar noord \\ 
Is uitgestrekte vlaktes onbewoon, \\ 
Waaroor die wind nog vrij waai, waar geen vlag, \\ 
In bloed gedoop, ons vrije lug besmet. \\ 
Daar brul die leeuw nog en die bulte beef, \\ 
Die wildsbok stamp nog wolke stof omhoog \\ 
En kwaggas hinnik week. \\ 
Die dood woed daar ook wel, dis waar, maar vrij, \\ 
En vat die slaap mij daar, slaap ek gerus, \\ 
Tevrede voort. Mog ek dan lê \\ 
Waar wije winde waai en vlaktes glooi, \\ 
Waar donder-echo’s oor die vlaktes sweep, \\ 
En eensaamheid mij laatste rus bewaak. \\ 
Dan sal ek droom en sal ek weet hoe stil \\ 
Die sonne sink daar deur die purper poort, \\ 
Waaruit die onbekende lok -- span in! \\ 
Laat die geratel van ons busse weer \\ 
Mij siel verkwik, die bre\"e skowwe nog \\ 
In slingerpas die vlaktes meet: ons trek \\ 
Reg noord, aan linkerhand die see, en regs, --  \\ 
Maar voor die ope vlaktes onbeknel. \\ 
\end{verse}
\attrib{(Met vergunning van ``De Burger'').}