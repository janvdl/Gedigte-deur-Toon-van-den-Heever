\poemtitle{Die skat van Petrossa}
\attrib{In 1837 is te Petrossa ’n skat ontdek, bestaande uit gou\"e bekers, met edelgesteente beset. \\ Die bekers moet uit verskillende lande afkomstig wees, blijkens die opskrifte. \\Die	skat dateer uit die Europese volksverhuising, en bestaat uit	die roof van Europa, die tribuut, wat een van ons Germaanse	voorvadere die volkere afgedwing het.}
\begin{verse}
\ \ \ Daar lê die skatte nou, hoop op hoop, \\ 
\ \ \ Daarvoor, was lande in bloed gedoop, \\ 
\ \ \ Geroof van die Sklaaf en swart Ethioop, \\ 
\ \ \ \ \ \ Hier is dit alles vergaar. \\ 
\end{verse}

\begin{verse}
Die kampvuur vlam op, glim oor vate van goud \\ 
\ \ \ En van silwer; karbonkels blits, ster aan ster, \\ 
\ \ \ En die vuurlig weerskijn in die agtergrond, ver, \\ 
Uit o\"e van gediertes wat staar uit die woud. \\ 
\end{verse}

\begin{verse}
\ \ \ Maar ek huiwer nog meer van die kil, kou\"e vuur, \\ 
\ \ \ Wat die Griekse slavin deur haar kijkers laat gluur, \\ 
\ \ \ Als sij sien, sij die kind van Athenes kultuur \\ 
\ \ \ \ \ \ Op die blonde barbaar. \\ 
\end{verse}

\begin{verse}
En sij sak in die dekens van purper en blouw, \\ 
\ \ \ Wat verward in die buit lê; die vuurlig bestraal \\ 
\ \ \ Arme wat blink soos ivoor; maar van staal \\ 
Is die o\"e, net so ferm, sonder vrees, sonder rouw. \\ 
\end{verse}

\begin{verse}
\ \ \ Die blonde veroweraar smeek en hij vlei, \\ 
\ \ \ Ontmoet net o\"e wat soos ijsskerwe snij: \\ 
\ \ \ Of hij bid, of hij dreig, moet die neerlaag maar lij, \\ 
\ \ \ \ \ \ Gaat vertwijfeld van daar. \\ 
\end{verse}

\begin{verse}
Die rooidag glim oor vate van goud \\ 
\ \ \ En silwer, karbonkels wat dof blink soos lood; \\ 
\ \ \ Op die purpere dekens, haalwit, net so koud, \\ 
Met haar dolk in haar hart lê die Griekse dood. \\ 
\end{verse}

\begin{verse}
\ \ \ Dit is sover als die storie lei; \\ 
\ \ \ Dit heet daar’s ’n ou, ou skat ontdek \\ 
\ \ \ In Petrossa, dateer uit die voorste trek; \\ 
\ \ \ Wat volg weet ek nie, ek belij. \\ 
\end{verse}

\begin{verse}
Tog ek sien vate, dekens van purper en blouw, \\ 
\ \ \ Wat verward in die buit lê, die vuurlig straal \\ 
\ \ \ Op arme wat skijn soos ivoor, -- maar van staal \\ 
Is die o\"e, net so koud, sonder skrik, sonder rouw. \\ 
\end{verse}
\attrib{(Met vergunning van ``De Burger'').}