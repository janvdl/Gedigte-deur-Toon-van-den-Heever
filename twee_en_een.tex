\poemtitle{Twee en Een}
\begin{verse}
\ \ \ \ \ \ \ \ \ \ \ \ \ \ Voor die ``ek'' nog bestaat \\
\ \ \ \ \ \ \ \ \ \ \ \ \ \ Word gesug en gesorg, \\
Verwagting met vreug word ontvange in wee; \\
\ \ \ \ \ \ \ \ \ \ \ \ \ \ Wat al is, word vir hoop \\
\ \ \ \ \ \ \ \ \ \ \ \ \ \ In die toekoms geborg, \\
Want die lewe is een, en die wil is vir twee. \\
\end{verse}

\begin{verse}
\ \ \ \ \ \ \ \ \ \ \ \ \ \ Lippies so lokkend \\
\ \ \ \ \ \ \ \ \ \ \ \ \ \ En blosse so sag, \\
Verleilik die ogies nou, dan weer verlee; \\
\ \ \ \ \ \ \ \ \ \ \ \ \ \ Dis nog skaars in die lewe, \\
\ \ \ \ \ \ \ \ \ \ \ \ \ \ Nog skaars uit die nag -- \\
Maar die lewe is een, en die wil is vir twee. \\
\end{verse}

\begin{verse}
\ \ \ \ \ \ \ \ \ \ \ \ \ \ Naar die kerk gaat dit hijgend \\
\ \ \ \ \ \ \ \ \ \ \ \ \ \ En wiegend en swaar, \\
Teen die boesem verlep moet die bijbeltjie leen; \\
\ \ \ \ \ \ \ \ \ \ \ \ \ \ Met vuur word gebid \\
\ \ \ \ \ \ \ \ \ \ \ \ \ \ En gevlei -- Ja dis waar, \\
Die lus is vir meer, maar die lewe’s net een. \\
\end{verse}