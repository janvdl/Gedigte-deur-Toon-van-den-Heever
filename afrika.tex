\poemtitle{Afrika}
\settowidth{\versewidth}{Soos die woedende golwe opbruis teen ’n rots dat dit beef in die diepte}
\begin{verse}[\versewidth]
Gevleugelde Telg van ou Tijd en jouw moeder, die mistiese ruimte, \\ 
Wijk uit mij verse, O spoed, en gun mij in rustige maat \\ 
Te sing van die lot van ons land, die wieg van die vroegste beskawing, \\ 
Vandaag nog die land van misterie en tog ook die land van die son; \\ 
Van die golwende vlakte, ons see, met sij deining en verblouw verskiet, \\ 
Met sij ewig diepsinnige lag soos die mond van ’n oosterse god. \\ 
\end{verse}

\begin{verse}[\versewidth]
Lang voor die geboorte van Mimer het Osiris, die Wijse, gelewe; \\ 
Toe Wodin, bekroon met orkane en bliksems, wat lik langs sij slape, \\ 
Eteries gedagtes bedenk, op die fluister van rawe gegrondves, \\ 
Het al die Songod sij pijle laat klink oor die snare van Memnon \\ 
En rustige wete geheers in die land van kolos en van sfinx. \\ 
Langs jouw kus het die skepe van Hiram geswalk, beskilder met o\"e, \\ 
Belaai met die skatte van Ofir, met wierook en elpebeen, \\ 
En houte, welriekend en keurig, geskenke vir Salomo’s Tempel, \\ 
Duisend-potig gevaartes met hulle rieme in ritmiese tred. \\ 
\end{verse}

\begin{verse}[\versewidth]
Soos die woedende golwe opbruis teen ’n rots dat dit beef in die diepte \\ 
En die sonlig ’n reenboog ontlok als dit skijn op die mistige skuim, \\ 
Net so het Barca, die trotse, verveel met haar skatte en vrede, \\ 
Soos ’n wolkbreuk Europa oorstelp en Rome geskud tot haar grond: \\ 
Die tijd het geknaag: Karthago en Rome is dinge van gister, \\ 
Maar die damp van die wrak en die botsing het luister versprei oor die \\ 
\ \ \ \ \ aarde. \\ 
\end{verse}

\begin{verse}[\versewidth]
Land van ewige weedom, Moeder van slawernij, \\ 
Wie sal beskrijf wat jij tors en die nood van jouw magtige bare! \\ 
Soos dit roer op die vlak van ’n vijwer dat golfies naar buite uitdije, \\ 
So het dit altijd beweeg in jouw donker, vernagte bestaan, \\ 
En volkere-golwe het een na die ander gedij en verpletter, \\ 
Daar blij van al die vergane beweging net bouwvalle oor \\ 
En die leeuwe vervul met geratel die saal van ’n naamlose god. \\ 
\end{verse}

\begin{verse}[\versewidth]
Maar wie kijk als die wolke-galleie gaan vaar oor die blouw van die \\ 
\ \ \ \ \ hemel, \\ 
Met haalwitte seile gebol, soos swane wat pronk in hul prag, \\ 
-- Dan wei in velde van goud die Gode hul vee oor die vlaktes --  \\ 
Voel hij dan nie hoe die Vrijheid gaat bruis in sij bloed, en sij senuws \\ 
Gaan tril, als hij denk met trots aan die wieg van die trekker-geslagte, \\ 
Waar die soekers naar ruimte die lug kon adem met tintlende bors? \\ 
\end{verse}

\begin{verse}[\versewidth]
Maar die droom is virbij en die hoop, O Afrika, moeder van smarte, \\ 
Wanneer is jouw beker geledig, of kom aan jouw weedom geen end? \\ 
Lij jij nog steeds aan die vloek wat Cham sij nakroos verwerf het, \\ 
Moet jij vir ewig die land van dienaars en houthakkers blij? \\ 
Gevleugelde telg van ou Tijd en jouw moeder, die mistiese ruimte, \\ 
Haas nog jouw vaart en verskeur die gordijn, wat die toekoms bewaar, \\ 
Verlaat die kaalhoofdige denkers, wat jouw afkoms bepeins en jouw \\ 
\ \ \ \ \ oorsprong, \\ 
Tot jij hul saam met die jare veeg in jouw vaderlik graf; \\ 
Toon ons die lig en die hoop met die rustige wete van more. \\ 
\end{verse}