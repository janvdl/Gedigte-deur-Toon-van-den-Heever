\poemtitle{Die beeld van Oom Paul}
\begin{verse}
Grouw in die maanlig rijs die swaar kolos, \\ 
\ \ \ Oorheers die jong geboomte aan sij voet, \\
\ \ \ Waardeur die aandwind huiwrig klaënd spoed, \\
En rustloos, als van geeste, sug die bos.
\end{verse}

\begin{verse}
Die maan maak daar ’n glimlig op sij wang, \\
\ \ \ Wat in die skemer lijk als nat betraan; \\
\ \ \ Hij peins oor jonggestorwe hoop, die waan \\
Van volkwees, peins oor volkswee en oor dwang.
\end{verse}

\begin{verse}
Voor sij gesig daar strek die doderijk, \\
\ \ \ Waar al sij tijd- en strijdgenote rus, \\
En aandwind saggies oor die slapers strijk.
\end{verse}

\begin{verse}
Wit-deinsrig skijn die maan oor stene alom.... \\
\ \ \ En grillig die verlede wat daar kijk \\
Op ’n gewese Afrikanerdom.
\end{verse}
\attrib{Met vergunning van ``De Burger''}