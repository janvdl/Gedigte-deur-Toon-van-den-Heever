\poemtitle{Madonna-diens}
\begin{verse}
Dit is ’n nag, die opstandig van drome, \\ 
\ \ \ \ \ En ons ontwaak uit slaap, wat ons rede noem, \\ 
En agter uit die skemer, uit die some, \\ 
\ \ \ \ \ Waar Digting Waarheid raak, kom opgedoem \\ 
\ \ \ \ \ \ \ \ \ \ Diana van die jag; die Mingodin, \\ 
\ \ \ \ \ \ \ \ \ \ \ \ \ \ \ Godinne vroeg aanbid; \\ 
\ \ \ \ \ Want Liefd’ moes lewe gee, die jag dit win: \\ 
\ \ \ \ \ \ \ \ \ \ Offers en kranse wit \\ 
Het altijd die altare van die twee getooi, \\ 
\ \ \ \ \ Geskenk van Jeugd wat sug naar roem, \\ 
\ \ \ \ \ \ \ \ \ \ Of skugter nooi.
\end{verse}

\begin{verse}
Die vrouwlikheid trek ons virewig aan. -- \\ 
\ \ \ \ \ Die Dondergod en Mars moes altwee swig, \\ 
Al het die Prins van Vrede vir ons voldaan, \\ 
\ \ \ \ \ \ \ \ \ \ \ \ \ \ \ Te luister was sij lig: \\ 
\ \ \ \ \ Madonna moes vereer word, Moedermaagd, \\ 
Haar slape moes bekrans met die kleurkring van die maan, \\ 
\ \ \ \ \ \ \ \ \ \ En deur die glorie van ’n grote God versaagd \\ 
Soek mens wat menslik is en bid Haar aan, \\ 
\ \ \ \ \ Bepeins haar pijn, haar wereldgrote smart \\ 
\ \ \ \ \ \ \ \ \ \ En weedom vir haar wig, \\ 
\ \ \ \ \ \ \ \ \ \ \ \ \ \ \ Haar moederhart.
\end{verse}

\begin{verse}
Die Eeuw is haastig, Liefste, maak gebaar \\ 
\ \ \ \ \ En in ons tijd van rede en ongeloof, \\ 
Verdeelde doele en twijfel, val ’n skaar \\ 
\ \ \ \ \ Van vrae ons aan en wil ons rus beroof. \\ 
\end{verse}

\begin{verse}
Die lug is swoel daarbuite, swijgend swaar \\ 
\ \ \ \ \ Van stroperige walm uit blommekelk, \\ 
Die kriek sing lente en die padde-skaar \\ 
\ \ \ \ \ Weergalm en swijg, die meibos, wit soos melk \\ 
Skijn in die maanlig wat in strome rus \\ 
\ \ \ \ \ Op die lente wat daar lok. \\ 
Soos hare van ’n vrouw wat saggies sus \\ 
\ \ \ \ \ En sing, haar kroos omkrans, \\ 
So druip die wilkertakkies loom en slank en glans \\ 
\ \ \ \ \ \ \ \ \ \ Soos silwergaas of elwerok \\ 
\ \ \ \ \ \ \ \ \ \ \ \ \ \ \ En rus.
\end{verse}