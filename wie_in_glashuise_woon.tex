\poemtitle{Wie in glashuise woon}
\begin{verse}
Bommelbij, bommelbij met jouw wit bandolier, \\ 
\ \ \ Met jouw vlerke wat blits in die sonlig en gons \\ 
\ \ \ En jouw swartpak, verguld met die blomme se dons, \\ 
Wat beteken jouw singe en wat maak jij hier? \\ 
\end{verse}

\begin{verse}
Maar die bommelbij werk, of ek vra en of nie, \\ 
\ \ \ Natuurlik, ek weet hij hoor nie naar mij \\ 
\ \ \ En laat mij gevra en gepeuter opsij, \\ 
Maar dis mos wat deurgaan vir po\"esie. \\ 
\end{verse}

\begin{verse}
Mij siel is so wijd als die oseaan, \\ 
\ \ \ En grootse gedagtes die woel daarin om, \\ 
\ \ \ Te groot vir woorde, mij Muse is stom, \\ 
Want waar haal sij sulke woorde vandaan? \\ 
\end{verse}

\begin{verse}
Ja, modder kan jij met jouw vingers nie vat, \\ 
\ \ \ Diamante is klein en gere\"eld van bouw \\ 
\ \ \ Als jouw o\"e maar reg is -- Ek vra vir jou, \\ 
Hoe moet ek die ruim van ’n siel dan skat? \\ 
\end{verse}

\begin{verse}
Dis newels, net newels wat warrel en vlie, \\ 
\ \ \ En ek is misskien nog agter die vlug \\ 
\ \ \ Van ou Tijd, en ken die boom bij sij vrug --  \\ 
Maar dis wat nou deurgaan vir po\"esie. \\ 
\end{verse}